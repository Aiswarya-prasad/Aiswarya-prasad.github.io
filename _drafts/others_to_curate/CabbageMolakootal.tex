Short description of what Cabbage molakootal is.

\large{\textbf{Ingredients}}
\begin{multicols}{2}
  \large{\textbf{Common}}
  \begin{itemize}
    \item Cabbage \hfill \textit{2 (Chopped)}
    \item Peas \textit{and/or} Carrot \textit{and/or} Green Beans \hfill \textit{x g}
  \end{itemize}
  \columnbreak
  \large{\textbf{Indian/special}}
  \begin{itemize}
    \item Toor dal \hfill \textit{X gram}
    \item Asafoetida \hfill \textit{1 pinch}
  \end{itemize}
\end{multicols}

\large{\textbf{Recipe}}

\begin{enumerate}
  \item Cook the toor dal in a pressure cooker (3 - 4 whistles) or in an open heavy-bottom pot (takes care not to let it burn at the bottom). The dal needs to be soft enough to easily make into a paste.
  \item Add Cabbage and other vegetable(s) to water.
  \item To the water add salt, turmeric powder, asafoetida.
  \item While water boils, mix coconut, cumin and dried red chillis in a jar and grind into thick a paste as smooth as possible.
  \item Add the paste to the boiling vegetables and allow to boil and incorporate.
  \item Add cooked toor to the boiling mixture and allow to thicken for 3 mins.
  \item Remove molakootal from heat.
  \item Finally, in a different small pan heat some oil and allow mustard seeds to splutter.
  \item To that add urad dal and as it begins to change colour, add asafoetida (optional) and some curry leaves.
  \item Empty on to molakootal and mix in before serving.
\end{enumerate}

\large{\textbf{Suggested pairing}}

Serve mixed in rice and with Potato/Colacasia/Brinjal/Raw banana/Yam fry on the side.
